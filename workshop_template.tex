\documentclass{article}

\title{CyberPaths\footnote{
\protectCopyright is held by the author/owner.
}\\
\vspace{0.2in}
\large Conference Workshop\\}

\author{
Xenia Mountrouidou\\
\affaddr{Department of Computer Science}\\
\affaddr{College of Charleston}\\
\email{mountrouidoux@cofc.edu}
}

\begin{document}

\maketitle

The goal of the project CyberPaths is the diversification and broadening of
the STEM talent pipeline in cybersecurity in predominantly undergraduate and
liberal arts schools. This is achieved by the creation of a curriculum that
accommodates students of different levels of computer literacy with focus on
experiential learning. This project mitigates the challenges undergraduate
institutions currently face in the cybersecurity area, for example, a tight
computer science curriculum and the inability to support the expensive
infrastructure required for cybersecurity education. To address these
challenges, first, we attract a diverse population of students by introducing
cybersecurity topics through multiple paths of study and engagement. Students
will be introduced to cybersecurity concepts through stand-alone course modules
and laboratory exercises injected in general education liberal arts courses.
Interested students can study further by taking two cybersecurity focused
courses and cybersecurity capstone projects created by this project. Second,
we use the Global Environment for Network Innovation (GENI) infrastructure in
the development of hands-on labs and the capstone project assignments. GENI
offers an affordable cloud solution to undergraduate institutions that lack the
infrastructure to support high overhead computer labs. In this talk, I will
present the CyberPaths project and briefly introduce the GENI labs and general
education modules that we have developed. Then we will complete a couple of
short GENI labs, starting from ``Hello GENI" and moving to a simple
``Denial of Service lab".

\end{document}
